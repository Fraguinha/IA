\section{Implementação}
\label{chap:estado-da-arte}

\subsection{Introdução}
\label{chap2:sec:intro}

Neste capítulo são apresentadas as bibliotecas utilizadas, as estruturas de dados escolhidas bem como todos os detalhes de implementação das respostas às questões.

\subsection{Dependências}
\label{chap2:sec:dependencias}

Ao longo da realização do trabalho foram utilizadas algumas bibliotecas \textit{built-in} e \textit{third-party} para auxilio à implementação da solução.

\begin{itemize}
  \item Bibliotecas \textit{built-in} do Python:
  \begin{enumerate}
    \item \textit{math}
    \item \textit{time}
  \end{enumerate}

  \item Bibliotecas \textit{third-party}:
  \begin{enumerate}
      \item \textit{networkx}~\cite{networkx}
  \end{enumerate}
\end{itemize}

\clearpage
\subsection{Representação da Informação}
\label{chap2:sec:info}

Durante a navegação do agente por parte do utilizador é registada informação, recolhida pelos sensores.

Foi necessário escolher estruturas de dados para representar esta informação de forma a facilitar a sua consulta e resposta às questões. As estruturas de dados escolhidas foram as seguintes:

\begin{itemize}
    \item \textbf{Grafo:} Foi utilizado para representar todos os quartos visitados. Uma \textit{edge} entre dois quartos indica uma porta e o seu \textit{weight} corresponde à distância euclidiana entre o centro dos dois quartos.
    \item \textbf{Dicionários:} Foram utilizados varios dicionários com o objectivo de obter consultas rápidas
    \begin{itemize}
        \item \textbf{Dicionário de Objetos Encontrados:} devolve uma lista de objectos encontrados num determinado quarto de uma certa categoria.
        \begin{itemize}
            \item \textbf{key}: (\textit{room}, \textit{category})
            \item \textbf{value}: \textit{object\_name}
        \end{itemize}
        \item \textbf{Dicionário de Tipos de Quarto:} devolve qual o tipo de um determinado quarto
        \begin{itemize}
            \item \textbf{key}: \textit{room}
            \item \textbf{value}: \textit{type\_of\_room}
        \end{itemize}
    \end{itemize}
    \item \textbf{Listas:} Foram utilizadas várias listas de forma a representar dados mais simples
    \begin{itemize}
      \item coordenadas centrais dos quartos
      \item numero dos corredores
      \item ...
    \end{itemize}
\end{itemize}

\clearpage
\subsection{Resposta às questões}
\label{chap2:sec:resposta}

Nesta secção é explicado de forma sucinta a implementação da solução para cada uma das perguntas sugeridas.

\subsubsection{Quantos quartos não estão ocupados?}
\label{chap2:subsec:q1}

Fazendo uso do grafo e dicionário de objetos encontrados (secção~\ref{chap2:sec:info}), iterou-se sobre todos os quartos visitados e contou-se em quantos destes se tinham observado pessoas.

Subtraindo os dois valores obtem se o número de quartos não visitados.


\subsubsection{Quantas suites foram encontradas até agora?}
\label{chap2:subsec:q2}

Esta questão foi respondida recorrendo ao dicionário de tipos de quartos (secção~\ref{chap2:sec:info}). Um quarto poderia ser classificado como suite em duas situações distintas:

\begin{enumerate}
    \item Se fosse observada uma cama nesse quarto e esse já se encontrasse ligado a outro quarto, que não um corredor, no Grafo.
    \item Se fosse adicionada uma \textit{edge} entre dois quartos (não corredores) e um desses quartos tivesse uma cama.
\end{enumerate}


\subsubsection{É mais provável encontrar pessoas nos corredores ou dentro dos quartos?}
\label{chap2:subsec:q3}

Considera-se que seria mais próvavel encontrar pessoas nos corredores se a proporção de corredor com pessoas fosse maior que a proporção de quartos com pessoas, e vice-versa.

Para isto efetuou-se a contagem de corredores visitados (\textit{visited\_corridors}), corredores com pessoas (\textit{corridors\_with\_people}), quartos visitados (\textit{visited\_rooms}), quartos com pessoas (\textit{rooms\_with\_people}).

Se:

\begin{equation}
    \frac{people\_corridors}{visited\_corridors} > \frac{people\_rooms}{visited\_rooms}
\end{equation}

Então será mais provável encontrar pessoas em corredores, caso contrário, será mais provável encontrar pessoas em quartos.


\subsubsection{Se quisesse encontrar um computador, a que tipo de quarto me devo dirigir?}
\label{chap2:subsec:q4}

Fazendo uso do dicionário de tipos de quartos e do dicionário de objetos observados por quarto, contamos o número de quartos de cada tipo e o número de quartos com computador por tipo de quarto.

Calcula-se seguidamente a proporção de quartos com computador para cada tipo de quarto e é apresentado ao utilizador o que apresenta maior proporção.


\subsubsection{Qual o número do quarto \textit{single} mais próximo?}
\label{chap2:subsec:q5}

Para responder a esta questão começou-se por obter uma lista com todos os quartos \textit{single} encontrados até ao momento.

Utilizou-se a função \textit{shortest\_path\_length}~\cite{shortest_path_lenght} da biblioteca \textit{networkx}~\cite{networkx} para calcular a distância da posição atual a cada um destes quartos.

Por fim verificou-se qual destes se encontra a menor distância.


\subsubsection{Como ir do quarto atual até ao elevador?}
\label{chap2:subsec:q6}

Nesta questão fez-se uso da função \textit{shortest\_path}~\cite{shortest_path} da biblioteca \textit{networkx}~\cite{networkx} para obter uma lista de quartos, desde o quarto atual (origem), até ao elevador (destino).

Por fim, foram apresentadas de forma intuitiva as direções ao utilizador.


\subsubsection{Quantos livros estima encontrar nos próximos 2 minutos?}
\label{chap2:subsec:q7}

Para responder a esta questão foram tidos em conta dois parâmetros:

\begin{enumerate}
    \item Número de quartos visitados por segundo (\textit{rooms\_sec})
    \item Número de livros observados por quarto (\textit{books\_room})
\end{enumerate}

Tendo em conta estes dois parâmetros, a estimativa do número de livros a encontrar nos próximos 2 minutos é dada por:

\begin{equation}
    prediction = (2 * 60) * rooms\_sec * books\_room
\end{equation}

É também tido em conta o número de quartos não visitados (\textit{remaining\_rooms}). Caso o número de quartos que é possível visitar em 2 minutos seja superior ao número de quartos não visitados, usa-se a equação:

\begin{equation}
    prediction = remaining\_rooms * books\_room
\end{equation}


\subsubsection{Qual a probabilidade de encontrar uma mesa num quarto sem livros e que tem pelo menos uma cadeira?}
\label{chap2:subsec:q8}

A probabilidade pedida é dada por: \cite{slides}

\begin{equation}
  P(T | \neg B \cap C) = \frac{P(T \cap \neg B \cap C)}{P(\neg B \cap C)}
\end{equation}

\begin{itemize}
    \item $T$ - Ter mesa
    \item $B$ - Ter livro
    \item $C$ - Ter cadeira
\end{itemize}

Para efetuar este cálculo contou-se o número de quartos que não tinham livros nem cadeiras, e quantos destes tinham mesas.

