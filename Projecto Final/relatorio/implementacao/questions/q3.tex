\subsubsection{É mais provável encontrar pessoas nos corredores ou dentro dos quartos?}
\label{chap2:subsec:q3}

Considera-se que seria mais próvavel encontrar pessoas nos corredores se a proporção de corredor com pessoas fosse maior que a proporção de quartos com pessoas, e vice-versa.

Para isto efetuou-se a contagem de corredores visitados (\textit{visited\_corridors}), corredores com pessoas (\textit{corridors\_with\_people}), quartos visitados (\textit{visited\_rooms}), quartos com pessoas (\textit{rooms\_with\_people}).

Se:

\begin{equation}
    \frac{people\_corridors}{visited\_corridors} > \frac{people\_rooms}{visited\_rooms}
\end{equation}

Então será mais provável encontrar pessoas em corredores, caso contrário, será mais provável encontrar pessoas em quartos.
