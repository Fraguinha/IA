\section{Reflexao Critica}
\label{chap:imp-test}

\subsection{Introdução}
\label{chap4:sec:intro}

Neste capítulo é analisado e discutido todo o trabalho realizado, de que forma este foi dividido pelos elementos do grupo e quais os problemas e dificuldades que surgiram.

\subsection{Divisão do Trabalho}
\label{chap4:sec:divisao}

As decisões relativas à escolha das estruturas de dados utilizadas foi feita em conjunto por ambos os elementos do grupo de forma a evitar a duplicação de informação.

Após ter sido decidido como representar a informação obtida, dividimos as questões em 2 grupos considerados de dificuldade equivalente e cada um dos elementos do grupo implementou o seu conjunto respectivo de questões.

\begin{enumerate}
  \item Set $A$: Resolvido por João Fraga
  \begin{itemize}
    \item questão 1 (Secção~\ref{chap2:subsec:q1})
    \item questão 3 (Secção~\ref{chap2:subsec:q3})
    \item questão 6 (Secção~\ref{chap2:subsec:q6})
    \item questão 7 (Secção~\ref{chap2:subsec:q7})
  \end{itemize}

  \item Set $B$: Resolvido por Ana Carolina Silva
  \begin{itemize}
      \item questão 2 (Secção~\ref{chap2:subsec:q2})
      \item questão 4 (Secção~\ref{chap2:subsec:q4})
      \item questão 5 (Secção~\ref{chap2:subsec:q5})
      \item questão 8 (Secção~\ref{chap2:subsec:q8})
  \end{itemize}
\end{enumerate}

Apesar desta divisão, todas as solução foram discutidas por ambos os elementos do grupo de forma a confirmar a sua correção.

\subsection{Problemas Encontrados}
\label{chap4:sec:problemas}

A navegação do agente era trabalhosa o que dificultava o teste das soluções, para além disso o simulador não era muito estável e requeria várias tentativas antes de funcionar correctamente.

Tendo em conta que o utilizador é responsável pela navegação do agente, por vezes poderá haver objetos que não são observados, levando à má classificação de quartos e erros na resposta às questões em geral.

Também devido à forma como o simulador está implementado, o agente consegue por vezes observar objectos através de paredes o que uma vez mais pode levar à má classificação de quartos e erros na resposta às questões em geral.

\subsection{Analise Crítica}
\label{chap4:sec:critica}

Apesar de terem sido atingidos todos os objectivos propostos, não foi feito uso de algumas das técnicas mais sofisticadas de Inteligência Artificial que teriam sido interessantes de explorar.
